\documentclass[sigconf, 10pt, review]{acmart}
% \documentclass[sigconf, 10pt, authordraft]{acmart}
\settopmatter{printfolios=true,printccs=false,printacmref=false}

\def\Title{Small and Versatile vs. Large and Specialized}
% \def\ShortTitle{}
\def\SubTitle{How to design a collection library for exploratory programming?}
\def\PdfTitle{\Title: \SubTitle}
\def\PdfSubject{TODO}
\def\PdfAuthors{{Stefan Marr, Benoit Daloze}}
\def\PdfKeywords{TODO}


%!TEX root = ../paper.tex

%% General Advice:
%% https://github.com/dspinellis/latex-advice/blob/master/README.md

%% paper independent setup

% \usepackage[square,numbers]{natbib}
\usepackage[utf8]{inputenc}
\usepackage{textgreek}
\usepackage{fixltx2e} % for \textsubscript
% \usepackage[authoryear,comma,square,sort&compress]{natbib}
\usepackage{ifpdf}
\usepackage{graphicx}

\newif\ifhtml
\ifpdf
\htmlfalse
\else
\htmltrue
\fi

\usepackage{scripts/collab}  % [hideall]
%!TEX root = ../paper.tex

\ifpdf
    \PassOptionsToPackage{bookmarks=true,bookmarksnumbered=true,
        pdfpagemode={UseOutlines},plainpages=false,pdfpagelabels=true,
        colorlinks=true,linkcolor={black},citecolor={black},urlcolor={black},
        pdftitle={\PdfTitle},%
        pdfsubject={\PdfSubject},%
        pdfauthor={\PdfAuthors},%
        pdfkeywords={\PdfKeywords},
        pdftex,unicode}{hyperref}

\renewcommand{\UrlBreaks}{\do\/\do\a\do\b\do\c\do\d\do\e\do\f\do\g\do\h\do\i\do\j\do\k\do\l\do\m\do\n\do\o\do\p\do\q\do\r\do\s\do\t\do\u\do\v\do\w\do\x\do\y\do\z\do\A\do\B\do\C\do\D\do\E\do\F\do\G\do\H\do\I\do\J\do\K\do\L\do\M\do\N\do\O\do\P\do\Q\do\R\do\S\do\T\do\U\do\V\do\W\do\X\do\Y\do\Z}

\else
    \ifhtml
        % \PassOptionsToPackage[%
        %    colorlinks=true,%
        %    linkcolor={black},%
        %    citecolor={black},%
        %    urlcolor={black},%
        %    tex4ht,unicode]{hyperref}
    \else
        \PassOptionsToPackage[%
           colorlinks=true,%
           linkcolor={black},%
           citecolor={black},%
           urlcolor={black}]{hyperref}
    \fi
\fi
% \usepackage{url}


% \input{scripts/listings-init.tex}
%% random stuff

\newcommand{\ie}{i.e.\xspace}
\newcommand{\eg}{e.g.\xspace}

\usepackage{letltxmacro}
\LetLtxMacro{\OrgCitep}{\citep}
\renewcommand{\citep}[1]{\,\OrgCitep{#1}}

\newcommand{\code}[1]{\lstinline!#1!}
\usepackage{listings}
\lstset{%
  basicstyle=\ttfamily
}

% define \citeurl
%  #1 link text (for HTML version)
%  #2 title
%  #3 author
%  [#4] date of retrival
%  #5  url
% Example:
%   \citeurl{linktext}{Welcome to the Jungle}{Herb Sutter}{27 June 2012}{http://herbsutter.com/welcome-to-the-jungle/}
% Result:
%   \footnote{\emph{Welcome to the Jungle}, Herb Sutter, access date: 27 June 2012
%     \url{http://herbsutter.com/welcome-to-the-jungle/}}
\usepackage{ifthen}
\newcommand{\citeurl}[5]{%
#1\footnote{\emph{#2}%
          \ifthenelse{\equal{#3}{}}%
                     {}%          then i.e. empty #3
                     {, #3}% else i.e. non-empty #3
          \ifthenelse{\equal{#4}{}}%
                     {}%          then i.e. empty #4
                     {, access date: #4}% else i.e. non-empty #4
, \url{#5}}}

\newcommand{\inlinequote}[1]{\emph{``#1''}}

\newcommand{\circled}[1]{\textcircled{\scriptsize\textsf{#1}}}

% \input{scripts/knitr-init.tex}

\usepackage[nameinlink]{cleveref}
\Crefformat{lstlisting}{#2Lst.\,#1#3}
\crefformat{lstlisting}{#2lst.\,#1#3}


%% Determine Latex Fonts:
% TEXTTT: \texttt{tt \expandafter\show\the\font tt}
% textbf \textbf{bf \expandafter\show\the\font bf}
% normal text { text \expandafter\show\the\font text }
% sans serif \textsf{sf \expandafter\show\the\font sf }

% ACMART:
%% texttt: font t1-zi4r-4.  -> Inconsolata
%% textbf: font LinLibertineTB-tlf-t1.
%% text:   font LinLibertineT-tlf-t1.
%% sans serif: font LinBiolinumT-tlf-t1



\collabAuthor{bd}{green!60!black}{Benoit}
\collabAuthor{sm}{red}{Stefan}

\usepackage{booktabs} % For formal tables

% Copyright
%\setcopyright{none}
%\setcopyright{acmcopyright}
%\setcopyright{acmlicensed}
\setcopyright{rightsretained}
%\setcopyright{usgov}
%\setcopyright{usgovmixed}
%\setcopyright{cagov}
%\setcopyright{cagovmixed}

\acmDOI{10.475/123_4} % TODO
\acmISBN{123-4567-24-567/08/06} % TODO

%Conference
\acmConference[PX/18]{Programming Experience Workshop}{April 2018}{Nice, France}
\acmYear{2018}
\copyrightyear{2018}

\acmPrice{15.00}
\acmSubmissionID{123-A12-B3}

\begin{document}
\title{\Title}
% \titlenote{Produces the permission block, and
%   copyright information}
\subtitle{\SubTitle}
% \subtitlenote{The full version of the author's guide is available as
%   \texttt{acmart.pdf} document}

\author{Stefan Marr}
\orcid{0000-0001-9059-5180}
\affiliation{
    \department{School of Computing}
    \institution{University of Kent}
    \city{Canterbury}
    \postcode{CT2 7NZ}
    \country{United Kingdom}
}
\email{s.marr@kent.ac.uk}

\author{Benoit Daloze}
\affiliation{
    \department{Institute for System Software}
    \institution{Johannes Kepler University Linz}
    \streetaddress{Altenbergerstraße, 69}
    \city{Linz}
    \postcode{4040}
    \country{Austria}
}
\email{benoit.daloze@jku.at}


% The default list of authors is too long for headers.
% \renewcommand{\shortauthors}{B. Trovato et al.}


\begin{abstract}
This paper provides a sample of a \LaTeX\ document which conforms,
somewhat loosely, to the formatting guidelines for
ACM SIG Proceedings.\footnote{This is an abstract footnote}
\end{abstract}

%
% The code below should be generated by the tool at
% http://dl.acm.org/ccs.cfm
% Please copy and paste the code instead of the example below.
%
% TODO
\begin{CCSXML}
<ccs2012>
 <concept>
  <concept_id>10010520.10010553.10010562</concept_id>
  <concept_desc>Computer systems organization~Embedded systems</concept_desc>
  <concept_significance>500</concept_significance>
 </concept>
 <concept>
  <concept_id>10010520.10010575.10010755</concept_id>
  <concept_desc>Computer systems organization~Redundancy</concept_desc>
  <concept_significance>300</concept_significance>
 </concept>
 <concept>
  <concept_id>10010520.10010553.10010554</concept_id>
  <concept_desc>Computer systems organization~Robotics</concept_desc>
  <concept_significance>100</concept_significance>
 </concept>
 <concept>
  <concept_id>10003033.10003083.10003095</concept_id>
  <concept_desc>Networks~Network reliability</concept_desc>
  <concept_significance>100</concept_significance>
 </concept>
</ccs2012>
\end{CCSXML}

% TODO
\ccsdesc[500]{Computer systems organization~Embedded systems}
\ccsdesc[300]{Computer systems organization~Redundancy}
\ccsdesc{Computer systems organization~Robotics}
\ccsdesc[100]{Networks~Network reliability}

% TODO
\keywords{ACM proceedings, \LaTeX, text tagging}


\maketitle

\begin{note}
TODO:
- Scala
  - has collection framework with mutable and immutable collections
    -> do they have the same structure 
    -> are immutable collections persistent?
    -> are APIs adapted to facilitate different style?
      -> perhaps not necessary since Scala embraces FP
  - uniform return type principle
  - \url{https://www.scala-lang.org/blog/2017/02/28/collections-rework.html}
 - \url{https://docs.scala-lang.org/overviews/core/architecture-of-scala-collections.html} 
   The Architecture of Scala Collections  

- Goldman Sachs Collections, now Eclipse Collections
\url{https://www.eclipse.org/collections/}
 - mutable and immutable collections
 - specializations for primitive types

- how do templates (expanding ones, as in C++) fit in here?
- compiler-level specialization as with miniboxing?

further reading:

- just in time data structure de wael et al.
- concurrency strategy paper

 - \url{https://www.microsoft.com/en-us/research/publication/bulk-types-with-class/}

 - \url{http://www.cs.ox.ac.uk/people/hongseok.yang/talk/2013-IP2/Lecture12.pdf}
   from \url{http://www.cs.ox.ac.uk/people/hongseok.yang/Public/Courses.html}
 
 - \url{http://web.cecs.pdx.edu/~black/publications/refactoringsACM.pdf}
   Applying Traits to the Smalltalk Collection Classes[/url]
 
 - Storage strategies for collections in dynamically typed languages \citep{Bolz:2013:SSC}

Interfaces and specifications for the Smalltalk-80 collection classes
\url{https://dl.acm.org/citation.cfm?id=141938}


There is also this for C++:
 \url{https://www.youtube.com/watch?v=1-CmNNp5eag}
 \url{http://stepanovpapers.com/stl.ppt}
... but it is ~15 years old.



Dominic:

Perhaps the following paper and thread of work might be relevant here:

* A Comparative Study of Language Support for Generic Programming, Garcia et al. OOPSLA 2003
\url{https://pdfs.semanticscholar.org/1cb3/a5ab80477e3f14e29a23b655a6acfb94957d.pdf}

This paper compared parameterised graph libraries in several languages with static polymorphism: (C++, SML, Haskell, Eiffel, Java, C\#). It was noted that C++ has the abilities to provide type-based refine implementations using the template mechanism, which Haskell could not do. This might be useful for you as it includes more of the OO side.

Perhaps relevant (though not in the OO space) is the following two papers which were motivated by the above:

* Associated Types with Class, Chakravarty et al. POPL 2005
\url{https://pdfs.semanticscholar.org/22d9/fa7835ae91047790dafcd61e30a7d9bd47bd.pdf}

and

* "Associated Type Synonyms" by Chakravarty et al. ICFP 2005.
\url{https://www.microsoft.com/en-us/research/wp-content/uploads/2005/01/at-syns.pdf}

These papers proposed and detailed a new type system feature for GHC/Haskell: primitive recursive type-level functions, motivated by features in C++ that made it more flexible for doing generic collections. I think the first few pages of each paper will be most useful as they explain how this can be used for doing generic collections, and also contain comparison with C++.

Papers potentially for:
 - Eiffel

@inproceedings{odersk2009,
	Address = {Dagstuhl, Germany},
	Author = {Martin Odersky and Adriaan Moors},
	Booktitle = {IARCS Annual Conference on Foundations of Software Technology and Theoretical Computer Science (FSTTCS 2009)},
	Editor = {Ravi Kannan and K Narayan Kumar},
	Pages = {427--451},
	Publisher = {Schloss Dagstuhl--Leibniz-Zentrum {f\"{u}r} Informatik},
	Series = {Leibniz International Proceedings in Informatics (LIPIcs)},
	Title = {Fighting Bit Rot with Types (Experience Report: Scala Collections)},
	Volume = {4},
	Year = {2009}}

@article{krogda2009,
	Author = {Stein Krogdahl and Birger M{\o}ller-Pedersen and Fredrik S{\o}rensen},
	Journal = {JoT},
	Month = {Nov},
	Number = {7},
	Pages = {59-85},
	Title = {Exploring the use of Package Templates for flexible re-use of Collections of related Classes},
	Volume = {8},
	Year = {2009}}
  
  
Newspeak related design questions (by Gilad)
is the map API compatible with single argument closures? It should be; are keys and values lists, and what order guarantees do we get for these, if any? Can lists be seen as maps with integer keys? Do we support slices? External vs. internal iteration? I feel I could keep going for a long time. One can even consider doing APL style vector operations - but that seems to me a whole different language.


Energy profiles of Java collections classes
https://dl.acm.org/citation.cfm?id=2884869

Reducing resource consumption of expandable collections: The Pharo case
- some data on collection use in Pharo
- the interesting bit seems to be on the implementation of Lua tables

https://www.sciencedirect.com/science/article/pii/S0167642317302940

Empirical Study of Usage and Performance of Java Collections
https://dl.acm.org/citation.cfm?id=3030221


CollectionSwitch: A Framework for Efficient and Dynamic Collection Selection
DOI 10.1145/3168825
CGO 2018 (to appear)
% \url{https://www.researchgate.net/publication/322438185_CollectionSwitch_A_Framework_for_Efficient_and_Dynamic_Collection_Selection






Dimensions of collections
- mutability (mutable, immutable)
- processing (sequential, parallel)
- persistence?

Discussion Points:
- which collections are actually used?
 -> there is a 2017 paper on it for Java
 -> ArrayList on top, there are some numbers for a Smalltalk paper
 

 - is small and versatile bad design because it violates the `one thing does one thing good premise'
   - it's a god object design smell/anti-pattern?
    \url{https://en.wikipedia.org/wiki/God_object}
    \todoref{anti-pattern book?}
    - do they do to much? is this an issue?
     -> yes, it's an issue, the code is not clear about what kind of 
        interactions one wants to allow on the collection
     -> complexity of implementation is a real issue (concurrency issues,
        strategies, JIT DS...)

Practical Implementation Concerns
 How to realize all this with same efficiency as when using special purpose
 collections?
 - probably not getting there 100\%, but 90\% or even 80\% are sufficient
   when it improves developer productivity (which still has to be proven of course)
  
 - to get to the 80-90\% mark
  -> strategies (seq, par)
  -> JIT Data Structures
     -> the other paper about collections I saw (something from a german university)
  ->

How to design so that specialized collections remain possible?
And are easily adopted?
 -> ensure subset of collection APIs are polymorphic
   -> ideally, one just changes the constructor, or adds a conversion
   -> idea would be that this gives the last 10-20\% of performance when needed
   -> leave door open for improving performance by restricting flexibility


Is this just an attempt to justify the complexity in the VM?

\end{note} 

% \section{Introduction}

The \textit{proceedings} are the records of a conference.\footnote{This
  is a footnote}  ACM seeks
to give these conference by-products a uniform, high-quality
appearance.  To do this, ACM has some rigid requirements for the
format of the proceedings documents: there is a specified format
(balanced double columns), a specified set of fonts (Arial or
Helvetica and Times Roman) in certain specified sizes, a specified
live area, centered on the page, specified size of margins, specified
column width and gutter size.

\section{The Body of The Paper}
Typically, the body of a paper is organized into a hierarchical
structure, with numbered or unnumbered headings for sections,
subsections, sub-subsections, and even smaller sections.  The command
\texttt{{\char'134}section} that precedes this paragraph is part of
such a hierarchy.\footnote{This is a footnote.} \LaTeX\ handles the
numbering and placement of these headings for you, when you use the
appropriate heading commands around the titles of the headings.  If
you want a sub-subsection or smaller part to be unnumbered in your
output, simply append an asterisk to the command name.  Examples of
both numbered and unnumbered headings will appear throughout the
balance of this sample document.

Because the entire article is contained in the \textbf{document}
environment, you can indicate the start of a new paragraph with a
blank line in your input file; that is why this sentence forms a
separate paragraph.

\subsection{Type Changes and {\itshape Special} Characters}

We have already seen several typeface changes in this sample.  You can
indicate italicized words or phrases in your text with the command
\texttt{{\char'134}textit}; emboldening with the command
\texttt{{\char'134}textbf} and typewriter-style (for instance, for
computer code) with \texttt{{\char'134}texttt}.  But remember, you do
not have to indicate typestyle changes when such changes are part of
the \textit{structural} elements of your article; for instance, the
heading of this subsection will be in a sans serif\footnote{Another
  footnote here.  Let's make this a rather long one to see how it
  looks.} typeface, but that is handled by the document class file.
Take care with the use of\footnote{Another footnote.}  the
curly braces in typeface changes; they mark the beginning and end of
the text that is to be in the different typeface.

You can use whatever symbols, accented characters, or non-English
characters you need anywhere in your document; you can find a complete
list of what is available in the \textit{\LaTeX\ User's Guide}
\cite{Lamport:LaTeX}.

\subsection{Math Equations}
You may want to display math equations in three distinct styles:
inline, numbered or non-numbered display.  Each of
the three are discussed in the next sections.

\subsubsection{Inline (In-text) Equations}
A formula that appears in the running text is called an
inline or in-text formula.  It is produced by the
\textbf{math} environment, which can be
invoked with the usual \texttt{{\char'134}begin\,\ldots{\char'134}end}
construction or with the short form \texttt{\$\,\ldots\$}. You
can use any of the symbols and structures,
from $\alpha$ to $\omega$, available in
\LaTeX~\cite{Lamport:LaTeX}; this section will simply show a
few examples of in-text equations in context. Notice how
this equation:
\begin{math}
  \lim_{n\rightarrow \infty}x=0
\end{math},
set here in in-line math style, looks slightly different when
set in display style.  (See next section).

\subsubsection{Display Equations}
A numbered display equation---one set off by vertical space from the
text and centered horizontally---is produced by the \textbf{equation}
environment. An unnumbered display equation is produced by the
\textbf{displaymath} environment.

Again, in either environment, you can use any of the symbols
and structures available in \LaTeX\@; this section will just
give a couple of examples of display equations in context.
First, consider the equation, shown as an inline equation above:
\begin{equation}
  \lim_{n\rightarrow \infty}x=0
\end{equation}
Notice how it is formatted somewhat differently in
the \textbf{displaymath}
environment.  Now, we'll enter an unnumbered equation:
\begin{displaymath}
  \sum_{i=0}^{\infty} x + 1
\end{displaymath}
and follow it with another numbered equation:
\begin{equation}
  \sum_{i=0}^{\infty}x_i=\int_{0}^{\pi+2} f
\end{equation}
just to demonstrate \LaTeX's able handling of numbering.

\subsection{Citations}
Citations to articles~\cite{bowman:reasoning,
clark:pct, braams:babel, herlihy:methodology},
conference proceedings~\cite{clark:pct} or maybe
books \cite{Lamport:LaTeX, salas:calculus} listed
in the Bibliography section of your
article will occur throughout the text of your article.
You should use BibTeX to automatically produce this bibliography;
you simply need to insert one of several citation commands with
a key of the item cited in the proper location in
the \texttt{.tex} file~\cite{Lamport:LaTeX}.
The key is a short reference you invent to uniquely
identify each work; in this sample document, the key is
the first author's surname and a
word from the title.  This identifying key is included
with each item in the \texttt{.bib} file for your article.

The details of the construction of the \texttt{.bib} file
are beyond the scope of this sample document, but more
information can be found in the \textit{Author's Guide},
and exhaustive details in the \textit{\LaTeX\ User's
Guide} by Lamport~\shortcite{Lamport:LaTeX}.

This article shows only the plainest form
of the citation command, using \texttt{{\char'134}cite}.

Some examples.  A paginated journal article \cite{Abril07}, an enumerated
journal article \cite{Cohen07}, a reference to an entire issue \cite{JCohen96},
a monograph (whole book) \cite{Kosiur01}, a monograph/whole book in a series (see 2a in spec. document)
\cite{Harel79}, a divisible-book such as an anthology or compilation \cite{Editor00}
followed by the same example, however we only output the series if the volume number is given
\cite{Editor00a} (so Editor00a's series should NOT be present since it has no vol. no.),
a chapter in a divisible book \cite{Spector90}, a chapter in a divisible book
in a series \cite{Douglass98}, a multi-volume work as book \cite{Knuth97},
an article in a proceedings (of a conference, symposium, workshop for example)
(paginated proceedings article) \cite{Andler79}, a proceedings article
with all possible elements \cite{Smith10}, an example of an enumerated
proceedings article \cite{VanGundy07},
an informally published work \cite{Harel78}, a doctoral dissertation \cite{Clarkson85},
a master's thesis: \cite{anisi03}, an online document / world wide web
resource \cite{Thornburg01, Ablamowicz07, Poker06}, a video game (Case 1) \cite{Obama08} and (Case 2) \cite{Novak03}
and \cite{Lee05} and (Case 3) a patent \cite{JoeScientist001},
work accepted for publication \cite{rous08}, 'YYYYb'-test for prolific author
\cite{SaeediMEJ10} and \cite{SaeediJETC10}. Other cites might contain
'duplicate' DOI and URLs (some SIAM articles) \cite{Kirschmer:2010:AEI:1958016.1958018}.
Boris / Barbara Beeton: multi-volume works as books
\cite{MR781536} and \cite{MR781537}.

A couple of citations with DOIs: \cite{2004:ITE:1009386.1010128,
  Kirschmer:2010:AEI:1958016.1958018}.

Online citations: \cite{TUGInstmem, Thornburg01, CTANacmart}.


\subsection{Tables}
Because tables cannot be split across pages, the best
placement for them is typically the top of the page
nearest their initial cite.  To
ensure this proper ``floating'' placement of tables, use the
environment \textbf{table} to enclose the table's contents and
the table caption.  The contents of the table itself must go
in the \textbf{tabular} environment, to
be aligned properly in rows and columns, with the desired
horizontal and vertical rules.  Again, detailed instructions
on \textbf{tabular} material
are found in the \textit{\LaTeX\ User's Guide}.

Immediately following this sentence is the point at which
Table~\ref{tab:freq} is included in the input file; compare the
placement of the table here with the table in the printed
output of this document.

\begin{table}
  \caption{Frequency of Special Characters}
  \label{tab:freq}
  \begin{tabular}{ccl}
    \toprule
    Non-English or Math&Frequency&Comments\\
    \midrule
    \O & 1 in 1,000& For Swedish names\\
    $\pi$ & 1 in 5& Common in math\\
    \$ & 4 in 5 & Used in business\\
    $\Psi^2_1$ & 1 in 40,000& Unexplained usage\\
  \bottomrule
\end{tabular}
\end{table}

To set a wider table, which takes up the whole width of the page's
live area, use the environment \textbf{table*} to enclose the table's
contents and the table caption.  As with a single-column table, this
wide table will ``float'' to a location deemed more desirable.
Immediately following this sentence is the point at which
Table~\ref{tab:commands} is included in the input file; again, it is
instructive to compare the placement of the table here with the table
in the printed output of this document.


\begin{table*}
  \caption{Some Typical Commands}
  \label{tab:commands}
  \begin{tabular}{ccl}
    \toprule
    Command &A Number & Comments\\
    \midrule
    \texttt{{\char'134}author} & 100& Author \\
    \texttt{{\char'134}table}& 300 & For tables\\
    \texttt{{\char'134}table*}& 400& For wider tables\\
    \bottomrule
  \end{tabular}
\end{table*}
% end the environment with {table*}, NOTE not {table}!

It is strongly recommended to use the package booktabs~\cite{Fear05}
and follow its main principles of typography with respect to tables:
\begin{enumerate}
\item Never, ever use vertical rules.
\item Never use double rules.
\end{enumerate}
It is also a good idea not to overuse horizontal rules.


\subsection{Figures}

Like tables, figures cannot be split across pages; the best placement
for them is typically the top or the bottom of the page nearest their
initial cite.  To ensure this proper ``floating'' placement of
figures, use the environment \textbf{figure} to enclose the figure and
its caption.

This sample document contains examples of \texttt{.eps} files to be
displayable with \LaTeX.  If you work with pdf\LaTeX, use files in the
\texttt{.pdf} format.  Note that most modern \TeX\ systems will convert
\texttt{.eps} to \texttt{.pdf} for you on the fly.  More details on
each of these are found in the \textit{Author's Guide}.

\begin{figure}
%\includegraphics{fly}
\caption{A sample black and white graphic.}
\end{figure}

\begin{figure}
%\includegraphics[height=1in, width=1in]{fly}
\caption{A sample black and white graphic
that has been resized with the \texttt{includegraphics} command.}
\end{figure}


As was the case with tables, you may want a figure that spans two
columns.  To do this, and still to ensure proper ``floating''
placement of tables, use the environment \textbf{figure*} to enclose
the figure and its caption.  And don't forget to end the environment
with \textbf{figure*}, not \textbf{figure}!

\begin{figure*}
%\includegraphics{flies}
\caption{A sample black and white graphic
that needs to span two columns of text.}
\end{figure*}


\begin{figure}
%\includegraphics[height=1in, width=1in]{rosette}
\caption{A sample black and white graphic that has
been resized with the \texttt{includegraphics} command.}
\end{figure}

\subsection{Theorem-like Constructs}

Other common constructs that may occur in your article are the forms
for logical constructs like theorems, axioms, corollaries and proofs.
ACM uses two types of these constructs:  theorem-like and
definition-like.

Here is a theorem:
\begin{theorem}
  Let $f$ be continuous on $[a,b]$.  If $G$ is
  an antiderivative for $f$ on $[a,b]$, then
  \begin{displaymath}
    \int^b_af(t)\,dt = G(b) - G(a).
  \end{displaymath}
\end{theorem}

Here is a definition:
\begin{definition}
  If $z$ is irrational, then by $e^z$ we mean the
  unique number that has
  logarithm $z$:
  \begin{displaymath}
    \log e^z = z.
  \end{displaymath}
\end{definition}

The pre-defined theorem-like constructs are \textbf{theorem},
\textbf{conjecture}, \textbf{proposition}, \textbf{lemma} and
\textbf{corollary}.  The pre-defined de\-fi\-ni\-ti\-on-like constructs are
\textbf{example} and \textbf{definition}.  You can add your own
constructs using the \textsl{amsthm} interface~\cite{Amsthm15}.  The
styles used in the \verb|\theoremstyle| command are \textbf{acmplain}
and \textbf{acmdefinition}.

Another construct is \textbf{proof}, for example,

\begin{proof}
  Suppose on the contrary there exists a real number $L$ such that
  \begin{displaymath}
    \lim_{x\rightarrow\infty} \frac{f(x)}{g(x)} = L.
  \end{displaymath}
  Then
  \begin{displaymath}
    l=\lim_{x\rightarrow c} f(x)
    = \lim_{x\rightarrow c}
    \left[ g{x} \cdot \frac{f(x)}{g(x)} \right ]
    = \lim_{x\rightarrow c} g(x) \cdot \lim_{x\rightarrow c}
    \frac{f(x)}{g(x)} = 0\cdot L = 0,
  \end{displaymath}
  which contradicts our assumption that $l\neq 0$.
\end{proof}

\section{Conclusions}
This paragraph will end the body of this sample document.
Remember that you might still have Acknowledgments or
Appendices; brief samples of these
follow.  There is still the Bibliography to deal with; and
we will make a disclaimer about that here: with the exception
of the reference to the \LaTeX\ book, the citations in
this paper are to articles which have nothing to
do with the present subject and are used as
examples only.
%\end{document}  % This is where a 'short' article might terminate



\appendix
%Appendix A
\section{Headings in Appendices}
The rules about hierarchical headings discussed above for
the body of the article are different in the appendices.
In the \textbf{appendix} environment, the command
\textbf{section} is used to
indicate the start of each Appendix, with alphabetic order
designation (i.e., the first is A, the second B, etc.) and
a title (if you include one).  So, if you need
hierarchical structure
\textit{within} an Appendix, start with \textbf{subsection} as the
highest level. Here is an outline of the body of this
document in Appendix-appropriate form:
\subsection{Introduction}
\subsection{The Body of the Paper}
\subsubsection{Type Changes and  Special Characters}
\subsubsection{Math Equations}
\paragraph{Inline (In-text) Equations}
\paragraph{Display Equations}
\subsubsection{Citations}
\subsubsection{Tables}
\subsubsection{Figures}
\subsubsection{Theorem-like Constructs}
\subsubsection*{A Caveat for the \TeX\ Expert}
\subsection{Conclusions}
\subsection{References}
Generated by bibtex from your \texttt{.bib} file.  Run latex,
then bibtex, then latex twice (to resolve references)
to create the \texttt{.bbl} file.  Insert that \texttt{.bbl}
file into the \texttt{.tex} source file and comment out
the command \texttt{{\char'134}thebibliography}.
% This next section command marks the start of
% Appendix B, and does not continue the present hierarchy
\section{More Help for the Hardy}

Of course, reading the source code is always useful.  The file
\path{acmart.pdf} contains both the user guide and the commented
code.

\begin{acks}
  The authors would like to thank Dr. Yuhua Li for providing the
  MATLAB code of the \textit{BEPS} method.

  The authors would also like to thank the anonymous referees for
  their valuable comments and helpful suggestions. The work is
  supported by the \grantsponsor{GS501100001809}{National Natural
    Science Foundation of
    China}{http://dx.doi.org/10.13039/501100001809} under Grant
  No.:~\grantnum{GS501100001809}{61273304}
  and~\grantnum[http://www.nnsf.cn/youngscientists]{GS501100001809}{Young
    Scientists' Support Program}.

\end{acks}


\bibliographystyle{ACM-Reference-Format}
\bibliography{references}

\end{document}
